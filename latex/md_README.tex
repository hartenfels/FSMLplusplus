F\-S\-M\-L implementation using C++ -- see \href{https://github.com/slebok/slepro/tree/master/languages/fsml}{\tt https\-://github.\-com/slebok/slepro/tree/master/languages/fsml}

\subsection*{Requirements }

This program requires a compiler that supports the C++11 standard.

It also depends on Boost, which is provided in the lib folder to be linked statically. There's no need to build Boost, as only header files are used.

For unit testing, Google's gtest library and pthread are required.

Finally, since the program generates La\-Te\-X and Dot code, you will need appropriate programs for compiling them, e.\-g. pdf\-Te\-X and Graphviz

\subsection*{Build }

There is a Makefile for gcc provided, which will place compiled binaries into a folder called bin.

Use {\ttfamily make debug} or {\ttfamily make release} to build a debug or fully-\/optimized executable. The binaries will be called {\ttfamily fsmlpp\-\_\-debug} and {\ttfamily fsmlpp} respectively.

Use {\ttfamily make simulation} to build the sample interactive simulation. If you have Google's gtest unit testing library and pthread installed. The resulting binary will be called {\ttfamily fsmlpp\-\_\-simulation}.

Use {\ttfamily make test} to build the unit test, the required libraries will be attempted to be linked via {\ttfamily -\/lgtest -\/lpthread} The resulting binary will be called {\ttfamily fsmlpp\-\_\-test}.

{\ttfamily make} by itself will build all targets.

\subsection*{Bash Script }

There is a bash script called {\ttfamily buildnrun.\-sh} provided. It will make the release build, generate C++, La\-Te\-X and Dot code for the Sample.\-Fsml, and finally run {\ttfamily pdflatex} and {\ttfamily graphviz} to yield two different P\-D\-F files.

\subsection*{Run F\-S\-M\-L++ }

The syntax for running the program is {\ttfamily fsmlpp S\-O\-U\-R\-C\-E}, where S\-O\-U\-R\-C\-E is the path to a file containing F\-S\-M\-L code.

Note that the program will look for the template files in the present working directory, so run it from the project's root directory.

The program will attempt to load the given file, parse the code inside, validate it and then generate a C++ header file, La\-Te\-X file and Dot file with an appropriate name. If any of the steps fail, an appropriate error is thrown.

\subsection*{Run Test }

Execute the compiled {\ttfamily fsmlpp\-\_\-test} from the project's root directory so it can find all files. No special output means that the expected errors or lack thereof occurred. Otherwise, there will be a message about each failed test.

\subsection*{Run Simulation }

To run the simulation, just execute the compiled {\ttfamily fsmlpp\-\_\-simulation} and enjoy the interactive simulation experience. 