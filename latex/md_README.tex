F\-S\-M\-L implementation using C++. See \href{https://github.com/slebok/slepro/tree/master/languages/fsml}{\tt https\-://github.\-com/slebok/slepro/tree/master/languages/fsml} for the reference implementation in Prolog.

\subsection*{Requirements}

This program requires a compiler that supports the C++11 standard.

It also depends on Boost, which is provided in the lib folder to be linked statically. There's no need to build Boost since the project exclusively uses header-\/only libraries.

For unit testing, Google's gtest library and pthread are required.

Finally, since the program generates La\-Te\-X and Dot code, you will need appropriate programs for compiling them into an image format, like pdf\-Te\-X and Graphviz.

\subsection*{Build}

There is a Makefile for gcc provided, which will place compiled binaries into a folder called bin.

Use {\ttfamily make debug} or {\ttfamily make release} to build a debug or fully-\/optimized executable. The binaries will be called {\ttfamily fsmlpp\-\_\-debug} and {\ttfamily fsmlpp} respectively.

Use {\ttfamily make simulation} to build the sample interactive simulation. If you have Google's gtest unit testing library and pthread installed. The resulting binary will be called {\ttfamily fsmlpp\-\_\-simulation}.

Use {\ttfamily make test} to build the test, the required libraries will be attempted to be linked via {\ttfamily -\/lgtest -\/lpthread} The resulting binary will be called {\ttfamily fsmlpp\-\_\-test}.

Use {\ttfamily make table} to build an executable that prints a table from the goedelization function for use in our presentation. The resulting binary will be called {\ttfamily fsmlpp\-\_\-table}.

{\ttfamily make} by itself will build all targets.

\subsection*{Bash Script}

There is a bash script called {\ttfamily buildnrun.\-sh} provided. It will make the release build, generate C++, La\-Te\-X and Dot code for the Sample.\-Fsml, and finally run {\ttfamily pdflatex} and {\ttfamily graphviz} to yield two different P\-D\-F files.

\subsection*{Run F\-S\-M\-L++}

The syntax for running the program is {\ttfamily fsmlpp S\-O\-U\-R\-C\-E}, where S\-O\-U\-R\-C\-E is the path to a file containing F\-S\-M\-L code.

Note that the program will look for the template files in the present working directory, so run it from the project's root directory.

The program will attempt to load the given file, parse the code inside, validate it and then generate a C++ header file, La\-Te\-X file and Dot file with an appropriate name. If any of the steps fail, an appropriate error is thrown.

\subsection*{Run Test}

The syntax for running the test is {\ttfamily fsmlpp\-\_\-test T\-E\-S\-T...}, where T\-E\-S\-T is one or more of the following\-:


\begin{DoxyItemize}
\item {\bfseries constraint} execute the simple constraints tests given.
\item {\bfseries model $<$ni$>$ $<$ns$>$ $<$t$>$ $<$max\-T$>$} execute identity testing of the parser, abstract syntax and flat representation. This will generate flat representations with {\bfseries ni} initial states, {\bfseries ns} non-\/initial states and transition configurations from {\bfseries t} to {\bfseries max\-T}. These will be converted to an abstract syntax tree, which is converted to F\-S\-M\-L code. This code is parsed into another abstract syntax tree and another flat representation is constructed from it. The abstract syntax trees and flat representation are compared to each other and should always be equal.
\item {\bfseries machine $<$ns$>$ $<$t$>$ $<$max\-T$>$} execute oracle testing of the machine against Boost's directed graph and, if the machine is actually valid, perform identity testing against the flat representation. This will generate flat representations together with an equivalent Boost.\-Graph directed graph with one initial state, {\bfseries ns} non-\/initial states and transition configurations from {\bfseries t} to {\bfseries max\-T}. A breadth-\/first search is performed on the graph. If all states are reachable, the flat representation is converted to a machine, which is converted back into another flat representation and the two are checked for equality. If not all states are reachable, it is verified that the construction of the machine fails with the same states reachable.
\item {\bfseries sanity\-Equal $<$ni$>$ $<$ns$>$ $<$t$>$} execute a simple equality sanity check. This will generate a flat representation and abstract syntax with {\bfseries ni} initial states, {\bfseries ns} non-\/initial states and a transition configuration {\bfseries t} and compare them against themselves.
\item {\bfseries sanity\-Inequal $<$ni1$>$ $<$ns1$>$ $<$t1$>$ $<$ni2$>$ $<$ns2$>$ $<$t2$>$} execute an inequality sanity check. This will generate two flat representations and abstract syntax trees with {\bfseries ni1} and {\bfseries ni2} initial states, {\bfseries ns1} and {\bfseries ns2} non-\/initial states and transition configurations {\bfseries t1} and {\bfseries t2} respectively. They are then compared against each other expecting them to be inequal. Note that the test will fail if two identical configurations are given.
\end{DoxyItemize}

Note that the {\bfseries ni} and {\bfseries ns} are limited by the size of {\ttfamily size\-\_\-t} on your platform, while {\bfseries t} and {\bfseries max\-T} are of arbitrary size.

\subsection*{Run Simulation}

To run the simulation, just execute the compiled {\ttfamily fsmlpp\-\_\-simulation} and enjoy the interactive simulation experience. 